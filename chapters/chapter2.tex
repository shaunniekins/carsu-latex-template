\chapter{REVIEW OF RELATED LITERATURE}

\section{Section 1}

Lorem ipsum dolor sit amet, consectetur adipiscing elit. Proin a ante vel nisi tincidunt condimentum.

\subsection{Sub-section 1}

Consequat cillum aliqua veniam velit irure elit nostrud qui fugiat excepteur consectetur consequat. Nulla labore magna aute nulla sunt deserunt fugiat sint ea reprehenderit tempor labore nostrud anim.

\subsubsection{Sub-sub-section 1.1}

Culpa duis labore enim tempor cupidatat et sunt. Veniam laborum ut sit qui nisi reprehenderit ut Lorem excepteur aliquip qui anim. Nisi consectetur esse dolore esse veniam excepteur ullamco fugiat est laboris.

\section{Section 2}

\subsection{Sub-section 2.2}

For example, many studies used approach X \parencite{placeholderArticle2023}, while others preferred approach Y \parencite{placeholderBook2023}.

\subsubsection{Sub-sub-section 1.1}

Culpa duis labore enim tempor cupidatat et sunt. Veniam laborum ut sit qui nisi reprehenderit ut Lorem excepteur aliquip qui anim. Nisi consectetur esse dolore esse veniam excepteur ullamco fugiat est laboris. Incididunt excepteur nulla dolor nisi nisi officia aliquip laboris occaecat officia in ea. Occaecat esse ad exercitation proident laboris eiusmod ut officia occaecat. Incididunt labore irure est non enim quis nulla consequat laborum mollit do labore anim. Deserunt ex dolor sint irure dolor tempor aute tempor minim irure do ullamco.

\section{Definition of Terms}

For clarity and consistency, the following key terms are defined as used in this thesis:

\begin{itemize}
  \item \textbf{Term 1:} Definition of the first key term. This definition might be adapted from a source \parencite{placeholderBook2023}.
  \item \textbf{Term 2:} Definition of the second key term. Explain how it is operationalized in your study.
  \item \textbf{Term 3:} Definition of the third key term. Clarify any specific nuances relevant to your research context.
\end{itemize}